
\documentclass{article}

\usepackage[T1]{fontenc}
\usepackage[utf8]{inputenc}
\usepackage[brazilian]{babel}
\usepackage[margin=0.5cm]{geometry}
\usepackage{titlesec}
\usepackage{titling}
\usepackage{hyperref}
\usepackage{MnSymbol}
\usepackage{longtable}
\usepackage{xpatch}
\usepackage{graphicx}
\usepackage{multicol}
% \setlength{\columnsep}{1.5cm}

\hypersetup{
    colorlinks=true,
    linkcolor=blue,
    filecolor=magenta,      
    urlcolor=cyan,
    pdftitle={Sharelatex Example},
    bookmarks=true,
    pdfpagemode=FullScreen,
}

\titlespacing*{\section}{0pt}{1.3em}{0.70em}
\titlespacing*{\subsection}{0pt}{0pt}{.1em}
\titlespacing*{\subsubsection}{0pt}{0.60em}{0.40em}

\titleformat{\section}
{\huge\bfseries\Large}
{}
{0em}
{}[\titlerule]

\titleformat{\subsection}
{\bfseries\large}
{}
{0em}
{}

\titleformat{\subsubsection}
{\bfseries}
{}
{0em}
{}

\author{Guilherme C. Q. Marthe}
\title{Currículo}

\renewcommand{\maketitle}{
\begin{flushleft}
	{\huge\theauthor}
	\vspace{0.2em}\\
	Av. Barão de Monte Mor, 225 apt 32 - São Paulo, SP 05687-010\\
	T (11) 3755-0719 --- C (11) 99161-7553\\
	GitHub: GuiMarthe | \href{mailto:gui.marthe@gmail.com}{gui.marthe@gmail.com}
\end{flushleft}
}

\newcommand{\jobdate}[3]{
\large
\vspace{0.05em} 
  {#1, #2 --- #3}
\vspace{0.5em} 
\\
}

\xpatchcmd{\itemize}
  {\def\makelabel}
  {\setlength{\itemsep}{0.2em}\def\makelabel}
  {}
  {}
\renewcommand\labelitemi{$\cdot$}

\begin{document}

\maketitle

\section{Perfil Profissional}

 \linespread{.99}\large{Economista formado pela PUC-SP com uma graduação em Estatística no Instituto de Matemática e Estatística da USP em andamento. Possui uma curiosidade natural em usar dados e modelagem para resolver problemas e estruturar projetos, bem como a paixão em expor as conclusões de suas soluções.  Possui seis anos de experiência com programação estatística, em R e quatro anos de construção de produtos alavancados por dados em Python. }

\vspace{0.1em}
%%%%%%%%%%%%%%%%%%%%%%%%%%%%%%%%%%%%%%%%%%%%%%%%

\section{Experiências e Principais Realizações}

\subsection{Cientista de Dados, Noverde}
\jobdate{São Paulo}{abril 2018}{Atualmente}
Na Noverde, uma fintech de crédito pessoal iniciada em 2016, trabalha conjuntamente com as lideranças da empresa a conduzindo área de dados para decisões automatizadas. É responsável pela concepção, análise, acompanhamento, decisões estratégicas e implementação do modelo de credit scoring que concedeu mais de 40 milhões de reais em 25 mil contratos. Internamente também canaliza esforços para automatização de tarefas, construções de pipelines de BI e realiza workshops internos para difusão de ferramentas e tecnologias que utiliza.

\subsubsection{Principais Realizações}

\begin{itemize}

\item{Com a implementação sucessiva do modelo de crédito da Noverde, conseguiu trazer a taxa de inadimplência de novos contratos de 42\% em março de 2018 para 22\% em dezembro de 2018;}
\item{Coordenou a construção de um modelo de scoragem customizado com o Serasa-Experian;}
\item{Construiu um protótipo de processamento de documentos que levou o custo de consultas de 4 reais para 20 centavos quando implementado;}
\item{Com o foco na experiência do cliente, construiu um protótipo de prova de vida de novos solicitantes que agilizaria o trabalho de análise de fraude a um baixo custo operacional;}
\item{Automatizou do acompanhamento de insumos de modelagem e fluxos populacionais para detectar mudanças bruscas o mais cedo possível;}

\end{itemize}


\subsection{Analista de dados, Umoe Bioenergy}
\jobdate{São Paulo e Presidente Prudente}{março 2017}{abril 2018}
Responsável pela gestão analítica dos dados da cadeia produtiva da companhia, reportando diretamente ao Diretor Financeiro e Presidente. Foco em identificar padrões e oportunidades de melhoria contínua através da análise de dados de várias etapas de produção agrícola como plantio, colheita, tratos culturais, manutenção de equipamentos, performance dos colaboradores, etc. Apresentação e discussão de análises e relatórios com gestores das áreas.

\subsubsection{Principais Realizações}

\begin{itemize}

\item{Criação de workflows e pipelines de manipulação de dados para automatizar relatórios de KPI’s em eficiência de maquinário e pessoal;}
\item{Por meio de mineração de dados, identificou operadores de maquinário underperformers encaminhando-os ao departamento pessoal que ficou responsável pelas remediações;}
\item{Utilizando-se de modelagem Bayesiana, identificou e quantificou gargalos na colheita mecanizada, auxiliando na tomada de decisão em investimentos de grande porte da usina;}

\end{itemize}

%%%%%%%%%%%%%%%%%%%%%%
\pagebreak
\subsection{Analista de produto, enjoei.com.br}
\jobdate{São Paulo, SP}{março 2016}{março 2017}
Parte da equipe que conceitualiza, específica e prioriza os projetos de novas funcionalidades e otimização da plataforma do Enjoei - um market-place concentrado na compra e venda de roupas usadas. Foi responsável direto por atender demandas de análises rigorosas utilizando aos dados estruturados e semi estruturados da equipe de produto, marketing e atendimento focando na tomada de decisão impulsionada por dados.
\subsubsection{Principais Realizações}

\begin{itemize}

\item{Implementar e otimizar o algoritmo base de ordenação dos produtos no site, com base em uma estimação de sellability de produtos em tempo real;}
\item{Implementar o primeiro modelo estatístico de recomendação de conglomerados de produtos baseado em vizinhanças, junto com o pipeline de previsão necessário para o seu uso;}
\item{Análise de padrões de pagamentos em boletos focado na redução da volatilidade desta variável;}
\item{Análise rigorosa de fatores importantes de receita e lucratividade de ações diretas no site, buscando o entendimento otimização no uso desses fatores;}
\item{Implementação da infraestrutura de dados necessárias para uma análise de tickets de atendimento do site;}
\item{Modelagem de receita prevista para diversos projetos com o intuito de estudar a viabilidade e a priorização deles;}

\end{itemize}

%%%%%%%%%%%%%%%%%%%%%%
\section{Informações Adicionais}
\xpatchcmd{\itemize}
  {\def\makelabel}
  {\setlength{\itemsep}{0em}\def\makelabel}
  {}
  {}


\setlength\multicolsep{0pt}

\begin{multicols}{2}
\normalsize

\subsection{Educação}
\begin{itemize}
\item{IME-USP – Grad. Estatística, em andamento} 
\item{Especialização em Data Science – JHU, interrompida} 
\item{Pontifícia Universidade Católica de São Paulo, Economia, 2014}
\item{Mount Vernon High School, Fairfax County VA, 2007/2008 (full year)}
\item{Expanish - Spanish School Buenos Aires}
\end{itemize}


\subsection{Conhecimentos técnicos}
\begin{itemize}
\item{Pacotes Estatísticos: R confortável; Python científico confortável; Spark básico} 
\item{Econometria de Séries Temporais (VAR, ARCH, ARIMA)} 
\item{Modelagem estatística aplicada: OLS, LMM, GLM, ANOVA, testes de hipóteses, bootstraping}
\item{Técnicas de ML: SVM, RF, LDA, KNN, KMC, PCA, t-SNE, e outras}
\item{Utilitários: SQL, Airflow, Jupyter, Docker, GCloud, git}
\end{itemize}


\subsection{Educação Complementar}
\begin{itemize}
\item{Machine Learning for Data Science and Analytics – ColumbiaX – Jan 2016} 
\item{Curso preparatório para ANPEC – Insper 2014 e 2015} 
\item{Simpósio de economia e Econometria dos preços das commodities, FGV EPGE - agosto 2012} 
\item{Operador de Mercado Financeiro Banker, Saint Paul, março a junho de 2011}
\end{itemize}

\columnbreak

\subsection{Publicações e apresentações em congressos}
\begin{itemize}
\item{Oliveira, Gesner; Heibel, Wagner; Marthe, Guilherme. Qualidade dos Serviços e Fatores da Independência das Agências Reguladoras: Uma Primeira Análise do Setor de Saneamento, Congresso ABAR, 2013} 
\item{Grupo de Economia da Infraestrutura e Soluções Ambientais. Como aumentar a concorrência e o investimento nos aeroportos, maio 2013.}
\end{itemize} 


\subsection{Referências}
\begin{itemize}
\item{Rakuten: Marcos Takashi, Engenheiro e Cientista de Dados \\C: +81 90-3684-4901} 
\item{GO Associados: Sr. Gesner Oliveira, Sócio Fundador\\gesner@goassociados.com.br C: (11) 98685-2715} 
\item{99Taxi: Leonardo Bianconi, Estratégia de Produto\\ C: (11) 98372-8107}
\end{itemize}

\end{multicols}




\end{document}

